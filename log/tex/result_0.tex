\documentclass[a4paper,12pt]{article}
\usepackage{cmap}
\usepackage[T2A]{fontenc}
\usepackage[utf8]{inputenc}
\usepackage[english, russian]{babel}
\usepackage{mathtext}
\usepackage{amsmath, amsfonts, amssymb, amsthm, mathtools}
\usepackage{textcomp}
\usepackage{euscript}
\usepackage{xcolor}
\usepackage{hyperref}
\usepackage[normalem]{ulem}
\definecolor{urlcolor}{HTML}{799B03}
\title{\textbf{Дифференцирование философов}}
\author{философ Эйден Манро}
\date{2023}
\begin{document}
\maketitle
\begin{center}
Мало создать шедевр. Надо еще найти людей, которые бы его оценили
\end{center}


\begin{center}
\begin{equation}
\cos(x^{((x + 2 \cdot x) + 1)})
\end{equation}
\end{center}

\begin{center}
Жизнь - это собачья упряжка: пока ты не лидер, пейзаж не меняется
\end{center}


\begin{center}
\begin{equation}
(\cos(x^{((x + 2 \cdot x) + 1)}))'
\end{equation}
\end{center}

\begin{center}
Красть мысли у одного человека - плагиат. Красть у многих - научное исследование
\end{center}


\begin{center}
\begin{equation}
x^{((x + 2 \cdot x) + 1)} \cdot (((1 + (0 \cdot x + 2 \cdot 1)) + 0) \cdot \ln(x) + ((x + 2 \cdot x) + 1) \cdot 1 \cdot  \frac{1 }{ x } )
\end{equation}
\end{center}

\begin{center}
Кто в армии служил, тот в цирке не смеется
\end{center}


\begin{center}
\begin{equation}
-1 \cdot \sin(x^{((x + 2 \cdot x) + 1)}) \cdot x^{((x + 2 \cdot x) + 1)} \cdot (((1 + (0 \cdot x + 2 \cdot 1)) + 0) \cdot \ln(x) + ((x + 2 \cdot x) + 1) \cdot 1 \cdot  \frac{1 }{ x } )
\end{equation}
\end{center}

\begin{center}
Красть мысли у одного человека - плагиат. Красть у многих - научное исследование
\end{center}


\begin{center}
\begin{equation}
-1 \cdot \sin(x^{((x + 2 \cdot x) + 1)}) \cdot x^{((x + 2 \cdot x) + 1)} \cdot (((1 + 2) + 0) \cdot \ln(x) + ((x + 2 \cdot x) + 1) \cdot  \frac{1 }{ x } )
\end{equation}
\end{center}

\end{document}
