\documentclass[a4paper,12pt]{article}
\usepackage{cmap}
\usepackage[T2A]{fontenc}
\usepackage[utf8]{inputenc}
\usepackage[english, russian]{babel}
\usepackage{mathtext}
\usepackage{amsmath, amsfonts, amssymb, amsthm, mathtools}
\usepackage{textcomp}
\usepackage{euscript}
\usepackage{xcolor}
\usepackage{hyperref}
\usepackage[normalem]{ulem}
\definecolor{urlcolor}{HTML}{799B03}
\title{\textbf{Дифференцирование древних викингов}}
\author{Эйден Манро мудрейший из викингов}
\date{2023}
\begin{document}
\maketitle
\begin{center}
Чистая совесть - лучшее снотворное
\end{center}

\begin{center}
\begin{equation}
(x \cdot \tan((x + \sin(x)))^{(x + \sin(x))})
\end{equation}
\end{center}

\begin{center}
Красть мысли у одного человека - плагиат. Красть у многих - научное исследование
\end{center}

\begin{center}
\begin{equation}
(\sin(x))'
\end{equation}
\end{center}

\begin{center}
Жизнь - это собачья упряжка: пока ты не лидер, пейзаж не меняется
\end{center}

\begin{center}
\begin{equation}
\cos(x) \cdot 1
\end{equation}
\end{center}

\begin{center}
Мудрый не тот, кто много размышляет о великом, а тот, кто мало думает о мелочах
\end{center}

\begin{center}
\begin{equation}
(\tan((x + \sin(x))))'
\end{equation}
\end{center}

\begin{center}
Чистая совесть - лучшее снотворное
\end{center}

\begin{center}
\begin{equation}
(\sin(x))'
\end{equation}
\end{center}

\begin{center}
Иногда только промахнувшись, понимаешь, как ты попал
\end{center}

\begin{center}
\begin{equation}
\cos(x) \cdot 1
\end{equation}
\end{center}

\begin{center}
Иногда только промахнувшись, понимаешь, как ты попал
\end{center}

\begin{center}
\begin{equation}
 \frac{1 }{ (\cos((x + \sin(x)))^{2}) }  \cdot (1 + \cos(x) \cdot 1)
\end{equation}
\end{center}

\begin{center}
Жизнь - это собачья упряжка: пока ты не лидер, пейзаж не меняется
\end{center}

\begin{center}
\begin{equation}
(x \cdot \tan((x + \sin(x)))^{(x + \sin(x))}) \cdot ((1 + \cos(x) \cdot 1) \cdot \ln(x \cdot \tan((x + \sin(x)))) + (x + \sin(x)) \cdot (1 \cdot \tan((x + \sin(x))) + x \cdot  \frac{1 }{ (\cos((x + \sin(x)))^{2}) }  \cdot (1 + \cos(x) \cdot 1)) \cdot  \frac{1 }{ x \cdot \tan((x + \sin(x))) } )
\end{equation}
\end{center}

\begin{center}
Чистая совесть - лучшее снотворное
\end{center}

\begin{center}
\begin{equation}
(x \cdot \tan((x + \sin(x)))^{(x + \sin(x))}) \cdot ((1 + \cos(x) \cdot 1) \cdot \ln(x \cdot \tan((x + \sin(x)))) + (x + \sin(x)) \cdot (1 \cdot \tan((x + \sin(x))) + x \cdot  \frac{1 }{ (\cos((x + \sin(x)))^{2}) }  \cdot (1 + \cos(x) \cdot 1)) \cdot  \frac{1 }{ x \cdot \tan((x + \sin(x))) } )
\end{equation}
\end{center}

\end{document}
