\documentclass[a4paper,12pt]{article}
\usepackage{cmap}
\usepackage[T2A]{fontenc}
\usepackage[utf8]{inputenc}
\usepackage[english, russian]{babel}
\usepackage{mathtext}
\usepackage{amsmath, amsfonts, amssymb, amsthm, mathtools}
\usepackage{textcomp}
\usepackage{euscript}
\usepackage{xcolor}
\usepackage{hyperref}
\usepackage[normalem]{ulem}
\definecolor{urlcolor}{HTML}{799B03}
\title{\textbf{Дифференцирование философов}}
\author{философ Эйден Манро}
\date{2023}
\begin{document}
\maketitle
\begin{center}
Чистая совесть - лучшее снотворное
\end{center}


\begin{center}
\begin{equation}
(\cos(x)^{\tan(x)})
\end{equation}
\end{center}

\begin{center}
Красть мысли у одного человека - плагиат. Красть у многих - научное исследование
\end{center}


\begin{center}
\begin{equation}
(\tan(x))'
\end{equation}
\end{center}

\begin{center}
Жизнь - это собачья упряжка: пока ты не лидер, пейзаж не меняется
\end{center}


\begin{center}
\begin{equation}
 \frac{1 }{ (\cos(x)^{2}) }  \cdot 1
\end{equation}
\end{center}

\begin{center}
Мудрый не тот, кто много размышляет о великом, а тот, кто мало думает о мелочах
\end{center}


\begin{center}
\begin{equation}
(\cos(x))'
\end{equation}
\end{center}

\begin{center}
Чистая совесть - лучшее снотворное
\end{center}


\begin{center}
\begin{equation}
-1 \cdot \sin(x) \cdot 1
\end{equation}
\end{center}

\begin{center}
Иногда только промахнувшись, понимаешь, как ты попал
\end{center}


\begin{center}
\begin{equation}
(\cos(x)^{\tan(x)}) \cdot ( \frac{1 }{ (\cos(x)^{2}) }  \cdot 1 \cdot \ln(\cos(x)) + \tan(x) \cdot -1 \cdot \sin(x) \cdot 1 \cdot  \frac{1 }{ \cos(x) } )
\end{equation}
\end{center}

\begin{center}
Иногда только промахнувшись, понимаешь, как ты попал
\end{center}


\begin{center}
\begin{equation}
(\cos(x)^{\tan(x)}) \cdot ( \frac{1 }{ (\cos(x)^{2}) }  \cdot \ln(\cos(x)) + \tan(x) \cdot -1 \cdot \sin(x) \cdot  \frac{1 }{ \cos(x) } )
\end{equation}
\end{center}

\end{document}
