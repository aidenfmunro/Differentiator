\documentclass[a4paper,12pt]{article}
\usepackage{cmap}
\usepackage[T2A]{fontenc}
\usepackage[utf8]{inputenc}
\usepackage[english, russian]{babel}
\usepackage{mathtext}
\usepackage{amsmath, amsfonts, amssymb, amsthm, mathtools}
\usepackage{textcomp}
\usepackage{euscript}
\usepackage{xcolor}
\usepackage{hyperref}
\usepackage[normalem]{ulem}
\definecolor{urlcolor}{HTML}{799B03}
\title{\textbf{Дифференцирование философов}}
\author{философ Эйден Манро}
\date{2023}
\begin{document}
\maketitle
\begin{center}
Красть мысли у одного человека - плагиат. Красть у многих - научное исследование
\end{center}


\begin{center}
\begin{equation}
\cos(15 \cdot x + 7)
\end{equation}
\end{center}

\begin{center}
Чистая совесть - лучшее снотворное
\end{center}


\begin{center}
\begin{equation}
(\cos(15 \cdot x + 7))'
\end{equation}
\end{center}

\begin{center}
Кто в армии служил, тот в цирке не смеется
\end{center}


\begin{center}
\begin{equation}
-1 \cdot \sin(15 \cdot x + 7) \cdot 0 \cdot x + 15 \cdot 1 + 0
\end{equation}
\end{center}

\begin{center}
Иногда только промахнувшись, понимаешь, как ты попал
\end{center}


\begin{center}
\begin{equation}
-1 \cdot \sin(15 \cdot x + 7) \cdot 15 + 0
\end{equation}
\end{center}

\end{document}
